\documentclass[12pt, letterpaper]{article}
\usepackage[utf8]{inputenc}

\hfuzz=1000em
\usepackage[letterpaper, portrait, margin=1in]{geometry}
\usepackage{setspace}
\doublespacing
\usepackage{graphicx}
\usepackage{hyperref}

\begin{document}

\begin{titlepage}
    \begin{center}

        \vspace*{5cm}

        \Large{\textbf{LOVR Demo Docs}}

        \vspace{-0.25cm}

        \Large{\textbf{for Interactive Music Experiences}}

        \vspace{0.5cm}

        \large{assembled by Zachary Talis}

        \vfill

        \small{Spring 2022}

    \end{center}
\end{titlepage}

\tableofcontents
\newpage

\section{Introduction}

\subsection{What does this doc cover?}
We're going to make a monkey spin in LOVR!\\
LOVR is a simple-yet-powerful game engine for VR. By the end of this doc, you'll have tackled these LOVR essentials:
\begin{itemize}
    \item Creating, UV-mapping, and exporting a simple 3D model from Blender.
    \item Slapping some materials onto the model in Substance Painter.
    \item Assembling a final texture in Krita.
    \item Writing a simple Lua script that LOVR uses to display our model.
\end{itemize}
Parts of this guide are based off LOVR's "Callbacks and Modules" documentation\footnote{\url{https://lovr.org/docs/Callbacks_and_Modules/}}.

\subsection{Tools we'll use}
You can download everything here for free!\\
\textbf{Blender}\footnote{\url{https://blender.org/}} is a 3D-modelling software that does a little bit of everything.\\
\textbf{Substance Painter}\footnote{\url{https://adobe.com/products/substance3d-painter.html}} is a nondestructive, mask-based, Adobe-owned texturing tool. It's free with an \verb|edu| email address.\\
\textbf{Krita}\footnote{\url{https://krita.org/}} is an image manipulation and painting tool.\\
\textbf{LOVR}\footnote{\url{https://lovr.org/}} is a cross-platform VR engine that flouts intuitive Lua scripting and a light footprint.

\section{Blender (Suzanneification)}

\subsection{A fresh Suzanne}
TK

\subsection{UVs for Suzanne}
TK

\section{Substance Painter}

\subsection{New file}
TK

\subsection{Rendering maps}
TK

\subsection{Smart materials}
TK

\subsection{Exporting}
TK

\section{Krita}

\subsection{An ambient excursion}
TK

\section{Blender (Final Export)}

\subsection{Replacing the old texture}
TK

\subsection{Exporting to GLTF}
TK

\section{LOVR}

\subsection{Project structure}
TK

\subsubsection{conf.lua, for convenience's sake}
TK

\subsection{Resource imports}
TK

\subsection{Spinny Suzanne}
TK

\subsection{Running the project}
TK

\end{document}